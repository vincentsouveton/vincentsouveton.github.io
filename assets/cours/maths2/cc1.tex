\documentclass{exam}[5pts]
\usepackage[left=1in, right=1in, top=1in, bottom=1in]{geometry}
\usepackage{amsmath}
\usepackage{nopageno}

\title{Examen de TD numéro 1 - Mathématiques S2}
\author{\small{Groupe CM2}}
\date{\small{14 février 2022}}

\begin{document}

\begin{center}
	\begin{Huge}
		Examen de TD numéro 1 - Mathématiques S2
	\end{Huge}
	
	\large V. Souveton
\end{center}

\vspace{2\baselineskip}

\begin{center}
\fbox{\fbox{\parbox{5.5in}{
\small{Cette épreuve dure 20 minutes. Il s'agit d'un QCM composé de 15 questions, noté sur 15 points. L'utilisation de documents, de la calculatrice ou de tout autre appareil électronique, est \textbf{interdite}. Pour répondre, il suffit d'entourer la lettre correspondante à ce que vous pensez être une bonne réponse. Chaque bonne réponse rapporte 1 point ; chaque mauvaise réponse enlève 0,5 point ; une absence de réponse n'enlève ni ne rapporte de point. Si votre note finale est négative, elle sera ramenée à $0$. Pour chaque question, il existe \textbf{une et une seule} bonne réponse ; ainsi, si votre réponse à une question comporte plusieurs choix, votre réponse sera considérée comme fausse. Bon courage !}}}}
\end{center}

\vspace{1\baselineskip}

\makebox[\textwidth]{Prénom et Nom : \hrulefill}

\vspace{0.75\baselineskip}

\makebox[\textwidth]{Note (/15) : \hrulefill}

\vspace{2\baselineskip}

\begin{questions}

\question La limite de $\frac{\exp(x)}{x}$ quand $x$ tend vers $+\infty$ :

\begin{oneparchoices}
 \choice n'existe pas
 \choice vaut $0$
 \choice vaut $1$
 \choice vaut $+\infty$
\end{oneparchoices}

\question La limite de $\frac{x}{\cos(x)}$ quand $x$ tend vers $+\infty$ :

\begin{oneparchoices}
 \choice n'existe pas
 \choice vaut $0$
 \choice vaut $1$
 \choice vaut $+\infty$
\end{oneparchoices}

\question La limite de $\frac{\ln (1+x)}{x}$ quand $x$ tend vers $0^+$ est :

\begin{oneparchoices}
 \choice $-\infty$ 
 \choice $0$
 \choice $1$
 \choice $+\infty$
\end{oneparchoices}

\question Quelle fonction est équivalente à $x \mapsto \sin(x)$ en 0 ?

\begin{oneparchoices}
 \choice $x \mapsto 1$
 \choice $x \mapsto x$ 
 \choice $x \mapsto x^2$
 \choice $x \mapsto \cos(x)$
  
\end{oneparchoices}
    
\question Quelle fonction est négligeable devant $x \mapsto \sin(x)$ en 0 ?

\begin{oneparchoices}
 \choice $x \mapsto 1$
 \choice $x \mapsto x$ 
 \choice $x \mapsto x^2$
 \choice $x \mapsto \cos(x)$
\end{oneparchoices}

\question Le DL$_5(0)$ de la fonction $\cos$ s'écrit :

\begin{choices}
 \choice $\cos(x) =  x -\frac{x^3}{6} + \frac{x^5}{120} + o(x^5)$ 
 \choice $\cos(x) =  1 + \frac{x^2}{2} + \frac{x^4}{24} + o(x^5)$
 \choice $\cos(x) =  1 - \frac{x^2}{2} + \frac{x^4}{24} + o(x^5)$
 \choice $\cos(x) =  1 - \frac{x^2}{2} + \frac{x^5}{120} + o(x^5)$
\end{choices}

\question Le DL$_3(0)$ de la fonction $f$ définie sur $\mathbf{R}$ par $f(x) = \exp(x)-1$ s'écrit :

\begin{choices}
 \choice $f(x) =  x + \frac{x^2}{2} + o(x^3)$ 
 \choice $f(x) =  x + \frac{x^2}{2} + \frac{x^3}{3} + o(x^3)$
 \choice $f(x) =  1 + x + \frac{x^2}{2} + \frac{x^3}{6} + o(x^3)$
 \choice $f(x) =  x + \frac{x^2}{2} + \frac{x^3}{6} + o(x^3)$
\end{choices}


\newpage


\question Le DL$_2(0)$ de la fonction $g$ définie sur $\mathbf{R}$ par $g(x) = \sin(2x)$ s'écrit :

\begin{choices}
 \choice $g(x) =  2x + o(x^2)$ 
 \choice $g(x) =  1 + \frac{x^2}{2} + o(x^2)$
 \choice $g(x) =  2x - \frac{8x^3}{6} + o(x^3)$
 \choice $g(x) =  x - \frac{8x^3}{6} + o(x^2)$
\end{choices}

\question Le DL$_0(0)$ de la fonction $h$ définie sur $\mathbf{R} \backslash \{ \frac{\pi}{2}+k\pi \ ; \ k \in \mathbf{Z} \}$ par $\displaystyle h(x) = \frac{\sin(x^4)-\left( \ln(1+x^2) \right)^2}{(1+x^2)\cos(x)}$ s'écrit :

\begin{choices}
 \choice $h(x) =  0$ 
 \choice $h(x) =  o(1)$
 \choice $h(x) =  x^6 + o(1)$
 \choice $h(x) =  x^6 + o(x^6)$
\end{choices}

\question La limite de $\frac{\sin(x)}{\ln(1+4x)}$ quand $x$ tend vers $0$ :

\begin{oneparchoices}
 \choice vaut $0$
 \choice vaut $0,25$
 \choice vaut $4$
 \choice n'existe pas
\end{oneparchoices}

\question Un équivalent simple de $\displaystyle \frac{\sqrt[3]{1+x^2}-\sqrt[2]{1-x^2}}{3x}$ en $0$ est donné par :

\begin{oneparchoices}
 \choice $0$
 \choice $\frac{-1}{3x}$
 \choice $\frac{\sqrt{x}}{3}$
 \choice $\frac{5x}{18}$
\end{oneparchoices}

\question Soit $h$ une fonction quatre fois dérivable sur $\mathbf{R}$ admettant le DL$_4(0)$ suivant : $h(x) =  2 + x - 3x^2 + 6x^4 + o(x^4)$. Dans ce cas, combien vaut $h''(0)$ ?

\begin{oneparchoices}
 \choice $-6$
 \choice $-3$
 \choice $3$
 \choice $6$
\end{oneparchoices}

\question Soit $f$ la fonction définie sur $\mathbf{R}\backslash\{-1\}$ par $f(x) = \frac{3}{1+x}$. Quelle est l'équation de la tangente à la courbe représentative de $f$ en $x=1$ ?

\begin{choices}
 \choice $y = \frac{3}{2} x - \frac{9}{4}$
 \choice $y = \frac{-3}{2} x + \frac{3}{2}$
 \choice $y = \frac{-3}{4} x + \frac{3}{2}$
 \choice $y = \frac{-3}{4} x + \frac{9}{4}$
\end{choices}

\question Soit $f$ la fonction définie sur $\mathbf{R}$ par $f(x) = \exp(-x^2)$. La tangente à la courbe représentative de $f$ en $0$ : 

\begin{choices}
 \choice est  en dessous de la courbe représentative de $f$ 
 \choice est au-dessus de la courbe représentative de $f$ 
 \choice traverse la courbe représentative de $f$ 
 \choice aucune des 3 propositions précédentes
\end{choices}

\question Soient deux fonctions $f$ et $g$ telles que $f \sim g$ en $+\infty$. Que peut-on directement en conclure ?

\begin{choices}
 \choice $\exp(f) \sim \exp(g)$ en $+\infty$ 
 \choice $\exp(f) = o(\exp(g))$ en $+\infty$ 
 \choice $\exp(g) = o(\exp(f))$ en $+\infty$ 
 \choice aucune des 3 propositions précédentes
\end{choices}

\end{questions}

\vspace{3\baselineskip}
\begin{center}
	\huge $\mathcal{FIN}$
\end{center}

\end{document}