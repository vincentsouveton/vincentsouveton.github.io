\documentclass[12pt,a4paper]{article}
\usepackage{amsfonts,amsmath,amsthm,amssymb,theorem}
\usepackage[french]{babel}
\usepackage{amsfonts,amsmath,amssymb,amsthm}
\usepackage[T1]{fontenc}
\usepackage{multicol}

\newif\ifcorrection
%\correctiontrue %commenter pour cacher la correction, laisser pour afficher la correction



\textwidth 17true cm \textheight 26.5true cm \hoffset=-2.cm
\voffset=-2.8cm
%\documentclass[12pt,a4paper]{article}
%\usepackage{amsfonts,amsmath,amssymb,}
%\usepackage{french}
%%\usepackage[francais]{babel}
%\input transfig
%\setlength{\parindent}{0pc}
%\setlength{\parskip}{0cm}
%\setlength{\oddsidemargin}{0cm}
%\setlength{\evensidemargin}{0cm}
%\setlength{\topmargin}{0pt}
%\setlength{\headheight}{0pt}
%\setlength{\textwidth}{6.5in}
%\setlength{\textheight}{10.0in}
%\setlength{\headsep}{0in}
%
\newcommand{\C}{{\mathbb{C}}}
\newcommand{\R}{{\mathbb{R}}}
\newcommand{\der}[1][f]{\left({\partial #1}\over{\partial x}\right)}
\newcommand{\K}{{\mathbb{K}}}
\newcommand{\N}{{\mathbb{N}}}
\newcommand{\Q}{{\mathbb{Q}}}
\newcommand{\Z}{{\mathbb{Z}}}
\newcommand{\im}{\mathop{\mathrm{Im}}\nolimits}
\newcommand{\id}{\mathop{\mathrm{Id}}\nolimits}
\renewcommand{\dim}{\mathop{\mathrm{dim}}\nolimits}
\renewcommand{\cosh}{\mathop{\mathrm{ch}}\nolimits}
\renewcommand{\sinh}{\mathop{\mathrm{sh}}\nolimits}
\renewcommand{\tanh}{\mathop{\mathrm{th}}\nolimits}
\newcommand{\argch}{\mathop{\mathrm{argch}}\nolimits}
\newcommand{\argsh}{\mathop{\mathrm{argsh}}\nolimits}
\newcommand{\argth}{\mathop{\mathrm{argth}}\nolimits}
\newcommand{\ds}{\displaystyle}
%%
%%
%%
%%>>>>>>>>>>>>>>>>>>>>>>>>>>>>>>>>>>>>>>>>>>>>>>>>>>>>>>>>>>>>>
%%
%%
\newcommand{\eto}{ ^{^{_{\scriptstyle{\ast}}}} }
\newcommand{\pardef}{\stackrel{d\acute ef}{=}}
%%
\newcommand{\Frac}[2]{\frac{\textstyle #1}
                          {\textstyle #2}}
\newcommand{\Sum} {\displaystyle \sum }
\newcommand{\Int} {\displaystyle \int }
\newcommand{\Prod} {\displaystyle \prod }
%%
\newcommand{\Mn}[1]{\mathcal{M}_n(#1)}
\newcommand{\ens}[2]{\lbrace #1,\,\cdots,\, #2 \rbrace}
%%
%%
%%
%%>>>>>>>>>>>>>>>>>>>>>>>>>>>>>>>>>>>>>>>>>>>>>>>>>>>>>>>>>>>>>
%%             EXERCICE
%%
%%
\newcounter{numexo}
\setcounter{numexo}{0}
\newenvironment{exo}[1][]%
{\stepcounter{numexo}%

\addvspace{\bigskipamount}%
\noindent\textbf{Exercice \thenumexo} : \textit{#1}\nopagebreak[4]%

\addvspace{\smallskipamount}\noindent }%
{%

\addvspace{\bigskipamount}%
}%
%%
\newenvironment{sol}[1][]%
{\addvspace{\bigskipamount}%
\noindent\textbf{Solution de l'exercice \thenumexo} : \textit{#1}\nopagebreak[4]%

\addvspace{\smallskipamount}\noindent }%
{%

\addvspace{\bigskipamount}%
}%
%%
\newcounter{exo}
\setcounter{exo}{0}
\newcommand{\exercice}{\addtocounter{exo}{1}
{\sc Exercice \theexo}\quad}

%\newtheorem{theo}{Th\'eor\`eme}[section]
%\newcommand{\rem}{{\bf Remarque :} }
%\newcommand{\proposition}[2]{\begin{#1}
 %                            \hfill\break
  %                           #2
   %                          \end{#1}
    %                        }
%\newtheorem{lem}{Lemme}[section]
%\newtheorem{prop}{Proposition}[section]

%\newtheorem{cor}{Corollaire}[section]
%\newtheorem{defi}{D\'efinition}[section]
%%
%%
%%>>>>>>>>>>>>>>>>>>>>>>>>>>>>>>>>>>>>>>>>>>>>>>>>>>>>>>>>>>>>>
% 	exposant dans le texte
\newcommand{\expost}[1]{\raisebox{1ex}{\scriptsize #1}}
%
%>>>>>>>>>>>>>>>>>>>>>>>>>>>>>>>>>>>>>>>>>>>>>>>>>>>>>>>>>>>>>
%
%
%Inclure une figure en centre%
\newcommand{\vcentre}[1]{\setbox1=\hbox{\input{#1}}
$\vcenter{\box1}$}
\newcommand{\vcentreps}[1]{\setbox1=\hbox{\epsfbox{#1}}
$\vcenter{\box1}$}
%%
%%
%%
%%
%%%%%%%%%%%%%%%%%%%%%%%%%%%%%%% begin{document} %%%%%%%%%%%%%%%%%%%%%%%%%%%%%%%%
%%
%%


\usepackage{color}
\usepackage{textcomp} % pour le caractre gros point
\definecolor{my-blue}{rgb}{0.2,0.4,1}


%%%%%%%%%%%%%%%%%%%%%%%%%%%%%%%%%%%%%%
\parindent=0pt

\def\Vect{{\rm Vect\,}}

\let\al=\alpha
\let\be=\beta
\let\ga=\gamma
\let\de=\delta
\let\la=\lambda
\begin{document}



Universit\'e Clermont Auvergne \hfill Ann\'ee 2021-2022\\
UFR de Math\'ematiques \hfill  


\begin{center}
\textsc{U.E. de Math\'ematiques S2}\medskip\\
Examen interm\'ediaire du mardi 22 mars 2022\\ 
%tbtb Dur\'ee : 1h30 minutes.

\end{center}

%%tbtb  debut
%tb \vspace{0.5cm}
%tb \textit{Les calculatrices, documents de cours, de travaux dirig\'es et les t\'el\'ephones sont interdits}
%tb \\
%tb \textit{Il sera tenu compte de la r\'edaction dans l'\'evaluation de la copie: toute r\'eponse non justifi\'ee n'apportera pas de point.}

%tb \hrulefill% \medskip\\
%tb \vspace{12mm}
\centerline{\rule{5cm}{1pt}}
%%
\begin{center}
%%
La dur\'ee de l'\'epreuve est de $1$ heure $30$.

Ce sujet comporte $1$ page.

Les documents, calculatrices, t\'el\'ephones et autres mat\'eriels \'electroniques sont interdits.

Il sera tenu compte de la r\'edaction dans l'\'evaluation, en particulier {\bf toute r\'eponse devra \^etre justifi\'ee avec soin} (aucun point ne sera donn\'e \`a une r\'eponse juste non justifi\'ee).

Bar\^eme pr\'evisionnel (pouvant \^etre modifi\'e) : Questions de cours 20$\%$, Ex. 1 30$\%$, Ex. 2 50$\%$
%%
\end{center}
%%
\centerline{\rule{5cm}{1pt}}


%%tbtb  fin


%%%%%%%%%%%%%%%%%%%%%


%%%%%%%%%%%%%%%%%%%%%


%%%%%%%%%%%%%%%%%%%%%

\vspace{0.5cm}

\noindent{\bf Questions de Cours.}
\begin{enumerate}

\item Soit $n\in \mathbb{N}$. \'Enoncer la formule de Taylor-Young \`a l'ordre $n$ en z\'ero pour une fonction $f$ d\'efinie et $n$ fois d\'erivable au voisinage de z\'ero.

\item Donner le d\'eveloppement limit\'e \`a l'ordre $5$ en $0$ de $h(x)=\dfrac1{1+x}$.

\item Soit $E$ un  espace vectoriel. Soient $F$ et $G$ deux sous-espaces vectoriels de $E$. 

D\'emontrer que $F\cap G$ est un sous-espace vectoriel de $E$.

\end{enumerate}



\vspace{0.3cm}

\begin{exo} 
%On rappelle que pour tout $x>0$ et tout $\alpha\in\R$ : $x^\alpha=\exp(\alpha\ln(x))$.

\begin{enumerate}

%tbtb debut
%tb \item Donner le domaine de d\'efinition $\mathcal{D}_f$ de la fonction $\displaystyle f(x)=\frac{\ln(1+x)}{(1+x)}$.
\item Donner le domaine de d\'efinition $\mathcal{D}_f$ de la fonction $f$ d\'efinie par : $\displaystyle f(x)=\frac{\ln(1+x)}{(1+x)}$.
%tbtb fin

\item D\'eterminer le d\'eveloppement limit\'e \`a l'ordre $3$ en $0$ de $f$.
  
%\item En d\'eduire le d\'eveloppement limit\'e \`a l'ordre $3$ en $0$ de la fonction $g: x\mapsto (1+x)^{\frac1{1+x}}$.

\item En d\'eduire les valeurs de $f^\prime(0)$, $f^{\prime\prime}(0)$ et $f'''(0)$.

\item D\'eterminer le d\'eveloppement limit\'e \`a l'ordre 3 de $g(x)=\cos(2x)-1$.
%\item D\'eterminer un \'equivalent simple de $x\mapsto \cos(x)-1$ en $0$.

\item En d\'eduire $\displaystyle\lim_{x\to 0}\frac{f(-x)+\sin(x)}{\cos(2x)-1}.$
\end{enumerate}

\end{exo}


\vspace{0.3cm}

\begin{exo} 
On consid\`ere les vecteurs de $\R^3$ \quad
$u_1=(1,-1,0),\quad u_2=(0,1,1)\quad \text{ et }\quad u_3=(2,1,3)$

et les sous-ensembles de $\R^3$ : 
$$F=\Vect(u_1,u_2,u_3),\quad G=\{(x,y,z)\in\R^3\ | \   x-2y+3z=0\}.$$


\begin{enumerate}

\item Justifier que $G$ est un sous-espace vectoriel de $\R^3$. 

\item D\'emontrer que $G$ est de dimension 2  et en donner une base $(v_1,v_2)$. 

\item Montrer que $(u_1,u_2)$ est une base de $F$.

\item Montrer que
$F=\{(x,y,z)\in\R^3\ | \ x+y-z=0\}.$

\item La famille $(u_1,u_2,v_1,v_2)$ est-elle libre? 

\item D\'eterminer une famille g\'en\'eratrice de $F+G$, puis d\'eterminer sa dimension.

\item En d\'eduire la dimension de $F\cap G$ et en donner une base.

\item Les sous-espaces vectoriels $F$ et $G$ sont-ils suppl\'ementaires dans $\R^3$?

\item D\'eterminer un sous-espace vectoriel suppl\'ementaire $H$ de $F$ dans $\R^3$.

\item D\'eterminer $u_F\in F$ et $u_H\in H$ tels que $u=(1,1,1)=u_F+u_H$. Cette \'ecriture est-elle unique?


\end{enumerate}


\end{exo}





\newpage


\begin{center}
	\huge Correction de l'examen intermédiaire
	
	\large Mathématiques S2 - 22 mars 2022
	
	\large V. Souveton
\end{center}

\vspace{1\baselineskip}

\section*{Questions de cours}

\begin{enumerate}
	\item $\displaystyle f(x) = f(0) + \sum_{k=1}^n \frac{f^{(k)}(0)}{k!}x^k + o(x^n)$, au voisinage de 0.
	
	\vspace{0.5\baselineskip}
	
	\item C'est un DL usuel : $\displaystyle \frac{1}{1+x} = 1 - x + x^2 - x^3 + x^4 - x^5 + o(x^5)$, au voisinage de 0.
	
	\vspace{0.5\baselineskip}
	
	\item D'abord, on a bien $F \cap G \subset E$.
	
	\vspace{0.25\baselineskip}
	
	\begin{itemize}
		\item $0_E \in F$ et $0_E \in G$ car $F$ et $G$ sont des sev de $E$. Donc $0_E \in F \cap G$.
		
		\vspace{0.25\baselineskip}
		
		\item Soient $x, y \in F \cap G$. Comme $x, y \in F$ et que $F$ est un sev de $E$, donc stable par addition, alors $x + y \in F$. De même, comme $x, y \in G$ et que $G$ est un sev de $E$, donc stable par addition, alors $x + y \in G$. En conclusion, $x+y \in F \cap G$. Donc $F \cap G$ est stable par addition.
		
		\vspace{0.25\baselineskip}
		
		\item Soient $x \in F \cap G$ et $\lambda \in \mathbf{R}$. Comme $x \in F$ et que $F$ est un sev de $E$, donc stable par produit externe, alors $\lambda \cdot x \in F$. De même, comme $x \in G$ et que $G$ est un sev de $E$, donc stable par produit externe, alors $\lambda \cdot x \in G$. En conclusion, $\lambda \cdot x \in F \cap G$. Donc $F \cap G$ est stable par produit externe.
	\end{itemize}

	\vspace{0.25\baselineskip}

	On a montré que $F \cap G$ est un sev de $E$.
	\vspace{0.5\baselineskip}
\end{enumerate}



\section*{Exercice 1}

\begin{enumerate}
	\item La fonction $f$ est définie pour $x+1 > 0$ et $x+1 \neq 0$, i.e. pour $x+1 > 0$. Donc $\mathcal{D}_f = ]-1,+\infty [$.
	
	\vspace{0.5\baselineskip}
	
	\item $f$ est le produit de $\ln(1+x)$ par $\frac{1}{1+x}$. Commençons par former le $DL_3(0)$ de $\ln(1+x)$, qui est usuel :
	\begin{align*}
			\ln(1+x) = x - \frac{x^2}{2} + \frac{x^3}{3} + o(x^3) \text{, quand } x \to 0.
	\end{align*}
	
	Puis, formons le $DL_3(0)$ de $1/(1+x)$, lui aussi usuel :
	\begin{align*}
		\frac{1}{1+x} = 1 - x + x^2 - x^3 + o(x^3) \text{, quand } x \to 0.
	\end{align*}
	
	Pour obtenir la partie régulière du $DL_3(0)$ de $f$, il suffit de multiplier la partie régulière du $DL_3(0)$ de $\ln(1+x)$ avec celle du $DL_3(0)$ de $1/(1+x)$, en omettant les termes d'ordre strictement plus grand que $3$. En conclusion,
	\begin{align*}
	f(x) = x - \frac{3}{2}x^2 +\frac{11}{6}x^3 + o(x^3) \text{, quand } x \to 0.
	\end{align*}
	
	\vspace{0.5\baselineskip}
	
	\item La fonction $f$ est 3 fois dérivable au voisinage de $0$. On peut donc lui appliquer la formule de Taylor-Young à l'ordre 3 en $0$ :
	\begin{align*}
		f(x) = f(0) + xf'(0) + x^2 \frac{f''(0)}{2} + x^3 \frac{f'''(0)}{6} + o(x^3) \text{, quand } x \to 0.
	\end{align*}
	
	Par unicité du DL, on peut identifier les coefficients de ce DL avec ceux obtenus à la question précédente. Cela donne : $f'(0) = 1$, $f''(0) = -3$ et $f'''(0) = 11$.
	
	\vspace{0.5\baselineskip}
	
	\item C'est immédiat pour peu que l'on se souvienne du $DL_3(0)$ de la fonction cosinus, qui est usuel : $\displaystyle \cos(2x) - 1 = 1 -\frac{(2x)^2}{2} - 1 + o(x^3) = -2x^2 + o(x^3)$, quand $x \to 0$.
	
	\vspace{0.5\baselineskip}
	
	\item En se rappelant que $\displaystyle \sin(x) = x -\frac{x^3}{6} + o(x^3)$ au voisinage de $0$, on obtient $\displaystyle f(-x) + \sin(x) = -\frac{3x^2}{2} - 2x^3 + o(x^3)$ au voisinage de $0$. Une fonction est équivalente en $0$ au premier terme non nul de son $DL(0)$ (s'il existe). Ici, le numérateur est donc équivalent à $-\frac{3x^2}{2}$. De même, le dénominateur équivaut à $-2x^2$. Par quotient d'équivalent, on en déduit que $\displaystyle \frac{f(-x) + \sin(x)}{\cos(2x)-1} \underset{x \to 0}{\sim} \frac{3}{4}$, qui tend vers $\displaystyle \frac{3}{4}$ quand $x$ tend vers $0$. Donc la limite cherchée est $\displaystyle \frac{3}{4}$.
	
	
\end{enumerate}




\thispagestyle{empty}

\section*{Exercice 2}

\begin{enumerate}
	\item On montre facilement que $G$ contient $0_{\mathbf{R}^3}$ et est stable par addition et produit externe.
	
	\vspace{0.5\baselineskip}
	
	\item $\displaystyle (x,y,z) \in G \Longleftrightarrow (x,y,z) = y \cdot (2,1,0) + z \cdot (-3,0,1)$. La famille de deux vecteurs $((2,1,0), (-3,0,1))$ est donc génératrice de $G$. Elle est clairement libre puisque les deux vecteurs qui la composent ne sont pas colinéaires. Donc c'est une base de $G$ et $\dim G = 2$.
	
	\vspace{0.5\baselineskip}
	
	\item La famille $(u_1,u_2,u_3)$ est génératrice de $F$, par définition du Vect. Or, $u_3 = 2u_1+3u_2$. Donc $(u_1,u_2)$ est génératrice de $F$. Puisqu'elle est libre, les deux vecteurs qui la composent n'étant pas colinéaires, alors $(u_1,u_2)$ est une base de $F$.
	
	\vspace{0.5\baselineskip}
	
	\item Appelons $K := \{ (x,y,z) \in \mathbf{R} \ | \ x+y-z=0\} = \{ (x,y,z) \in \mathbf{R} \ | \ y=-x+z\}$. 
	
	\vspace{0.25\baselineskip}
	
	Alors, $\displaystyle (x,y,z) \in K \Longrightarrow (x,y,z) = x \cdot (1,-1,0) + z \cdot (0,1,1) = x \cdot u_1 + y \cdot u_2 \Longrightarrow (x,y,z) \in \Vect(u_1,u_2) = F$. Ainsi, $K \subset F$. 
	
	\vspace{0.25\baselineskip}
	
	Aussi, $u_1, u_2 \in K$ (ils vérifient l'équation de F). Puisque K est un sev de $\mathbf{R}^3$, donc stable par CL, alors toute combinaison linéaire de $u_1$ et de $u_2$, i.e. tout vecteur de $F$, est aussi dans $K$. Ainsi, $F \subset K$. 
	
	\vspace{0.5\baselineskip}
	
	Par double inclusion, on a donc prouvé que $F = K$.
	
	\vspace{0.5\baselineskip}
	
	\item C'est une famille de 4 vecteurs dans un espace vectoriel de dimension 3, donc elle ne peut pas être libre. En particulier, $u_1-3u_2+4v_1+3v_2 = 0_{\mathbf{R}^3}$ donc $v_2$ est CL des trois autres vecteurs de la famille.
	
	\vspace{0.5\baselineskip}
	
	\item $\displaystyle F+G = \Vect(u_1,u_2)+ \Vect(v_1,v_2) = \Vect(u_1,u_2,v_1,v_2) = \Vect(u_1,u_2,v_1)$. La famille $(u_1,u_2,v_1)$ est donc génératrice de $F+G$. On vérifie aisément qu'elle est libre, ce qui en fait une base de $F+G$. Ainsi, $\dim(F+G) = 3$.
	
	\vspace{0.5\baselineskip}
	
	\item On applique la formule de Grassmann : $\displaystyle \dim (F+G) = \dim(F)+\dim(G)-\dim(F\cap G)$ et on trouve $\displaystyle \dim F \cap G = 1$. Il suffit donc de trouver un vecteur non nul appartenant à la fois à $F$ et à $G$. Par exemple, le vecteur $(-1,4,3) \in F \cap G$. Donc $((-1,4,3))$ est une base de $F \cap G$.
	
	\vspace{0.5\baselineskip}
	
	\item $\dim F \cap G = 1$ donc $F \cap G \neq \{(0,0,0)\}$. Ainsi, $F$ et $G$ ne sont pas en somme directe, donc pas supplémentaires.
	
	\vspace{0.5\baselineskip}
	
	\item Pour cela, on commence par compléter la base $(u_1, u_2)$ de $F$ en une base de $\mathbf{R}^3$. Il manque un seul vecteur, que nous allons choisir parmi les vecteurs de la base canonique de $\mathbf{R}^3$. Par exemple, la famille $(u_1, u_2, e_2)$, avec $e_2 := (0,1,0)$, est libre et contient $3 = \dim \mathbf{R}^3$ vecteurs, donc c'est une base de $\mathbf{R}^3$. Un supplémentaire de $F$ dans $\mathbf{R}^3$ est donc $H = \Vect(e_2)$.
	
	\vspace{0.5\baselineskip}
	
	\item $\displaystyle u = (u_1 + u_2) + e_2$, avec $(u_1 + u_2) \in F$ et $e_2 \in H$. Cette écriture est unique, les espaces $F$ et $H$ étant supplémentaires.
\end{enumerate}


\vspace{2\baselineskip}

\begin{center}
	\huge $\mathcal{FIN}$
\end{center}










\end{document}


	
