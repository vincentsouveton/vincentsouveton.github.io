\documentclass[french]{article}
\usepackage[T1]{fontenc}
\usepackage[utf8]{inputenc}
\usepackage{lmodern}
\usepackage[a4paper]{geometry}
\geometry{hmargin=2.5cm,vmargin=1.5cm}
\usepackage{babel}
\usepackage{amsmath,amsfonts,amssymb}

\title{Controle continu de TD numéro 2}
\author{Groupe CM2}
\date{25 mars 2022}

\begin{document}
\thispagestyle{empty}

\begin{center}
	\Large Contrôle continu de TD numéro 3 
	
	\large Mathématiques S2 - 14 avril 2022
	
	\large V. Souveton
\end{center}

\vspace{0.5\baselineskip}

\noindent
On se place dans l'espace vectoriel $\mathbf{R}^3$ muni de la somme et du produit externe usuels. On appelle $e_1, e_2$ et $e_3$ les 3 vecteurs de la base canonique ($e_i$ est le vecteur dont toutes les coordonnées sont nulles sauf la $i$-ème, qui vaut 1). On considère l'endomorphisme $f : \mathbf{R}^3 \to \mathbf{R}^3$ défini par :
\begin{equation*}
	f(x,y,z) = (3x+y,0,x+y+z)
\end{equation*}

\begin{enumerate}
	\item Montrer que $f$ est linéaire. \textbf{(/1)}
	
	\vspace{0.5\baselineskip}
	
	\item Écrire la matrice $M$ de $f$ dans la base canonique de $\mathbf{R}^3$. Calculer le déterminant de cette matrice ; cette matrice est-elle inversible ? \textbf{(/2)}
	
	\vspace{0.5\baselineskip}
	
	\item Déterminer le noyau de $f$, en donner une base et préciser sa dimension. \textbf{(/2)}
	
	\vspace{0.5\baselineskip}
	
	\item Rappeler le théorème du rang dans le cas général. Ici, combien vaut la dimension de Im $f$ ? \textbf{(/1,5)}
	
	\vspace{0.5\baselineskip}
	
	\item Donner une base de Im $f$. \textbf{(/1)}
	
	\vspace{0.5\baselineskip}
	
	\item L'application $f$ est-elle bijective ? Injective ? Surjective ? \textbf{(/0,5)}
	
	\vspace{0.5\baselineskip}
	
	\item On pose $u_1 := (0,0,1)$, $u_2 := (2,0,1)$ et $u_3 := (1,-3,2)$.
	
	\vspace{0.25\baselineskip}
	
		\subitem a) Montrer que $(u_1,u_2,u_3)$ est une base de $\mathbf{R}^3$. \textbf{(/0,5)}
		
		\subitem b) Écrire la matrice $N$ de $f$ dans la base $(u_1,u_2,u_3)$. \textbf{(/1,5)}
		
\end{enumerate}

\vspace{2\baselineskip}

\begin{center}
	\huge $\mathcal{FIN}$
\end{center}

\end{document}