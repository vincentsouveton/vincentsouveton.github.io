\documentclass[french]{article}
\usepackage[T1]{fontenc}
\usepackage[utf8]{inputenc}
\usepackage{lmodern}
\usepackage[a4paper]{geometry}
\geometry{hmargin=2.5cm,vmargin=1.5cm}
\usepackage{babel}

\title{Controle continu de TD numéro 2}
\author{Groupe CM2}
\date{25 mars 2022}

\begin{document}
\thispagestyle{empty}

\begin{center}
	\huge Contrôle continu de TD numéro 2 
	
	\large Mathématiques S2 - 25 mars 2022 - Groupe CM2
	
	\large V. Souveton
\end{center}

\vspace{1\baselineskip}

\noindent
On se place dans l'espace vectoriel $E := \mathcal{F}(\mathbf{R},\mathbf{R})$ des fonctions de la variable réelle à valeurs dans l'ensemble des réels, muni de la somme et du produit externe usuels. On considère : 

\begin{itemize}
	\item $P := \{ f \in \mathcal{F}(\mathbf{R},\mathbf{R}) \ | \ \forall x \in \mathbf{R}, \ f(-x) = f(x)\}$, l'ensemble des fonctions paires ;
	\item $I := \{ f \in \mathcal{F}(\mathbf{R},\mathbf{R}) \ | \ \forall x \in \mathbf{R}, \ f(-x) = -f(x)\}$, l'ensemble des fonctions impaires.
\end{itemize}

\vspace{1\baselineskip}

\begin{enumerate}
	\item Donner trois exemples de fonctions paires et trois exemples de fonctions impaires. \textbf{(/3)}
	
	\vspace{1\baselineskip}
	
	\item Montrer que $P$ et $I$ sont des sous-espaces vectoriels de $E$. \textbf{(/5)}
	
	\vspace{1\baselineskip}
	
	\item Montrer que $P$ et $I$ sont en somme directe. \textbf{(/2)}
	
	\vspace{1\baselineskip}
	
	\item Pour tout $f \in E$, on considère la fonction $p_f$ définie pour tout réel $x$ par $p_f(x) = f(x) + f(-x)$ et la fonction $i_f$ définie pour tout réel $x$ par $i_f(x) = f(x) - f(-x)$.
	
	\vspace{0.5\baselineskip}
	
		\subitem a) Montrer que pour tout $f \in E$, la fonction $p_f$ est paire. \textbf{(/1,5)}
		\subitem b) Montrer que pour tout $f \in E$, la fonction $i_f$ est impaire. \textbf{(/1,5)}
		\subitem c) En déduire que $P$ et $I$ sont supplémentaires dans $E$. \textbf{(/2)}
\end{enumerate}

\vspace{2\baselineskip}

\begin{center}
	\huge $\mathcal{FIN}$
\end{center}

\end{document}